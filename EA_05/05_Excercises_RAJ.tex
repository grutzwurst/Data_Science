\documentclass[11pt,a4paper]{article}
\usepackage[T1]{fontenc}
\usepackage[ngerman]{babel}
\usepackage{amsmath}
\usepackage{parskip}
%\usepackage{graphicx}
\usepackage{listings}
\usepackage{hyperref}

%opening
\author{Simon Cholewa}
\title{5. R and Julia Excercise}

\hyphenation{
	Mo-tor-ü-ber-wach-ung 
}


\begin{document}

\maketitle

\section{R}
\textit{Write a program to guess a number in between 0 and 100!}

\textit{Hence the computer invents the number and the user = you tries to guess it! }

\lstinputlisting[numbers=left, language=R, frame=single, tabsize=2, basicstyle=\footnotesize, showstringspaces=false]{Zahlenraten.R}

\newpage


\section{R}
\textit{Analyse the \textbf{esoph} dataset. Can you derive some useful statements from it? Use data() to see all available datasets. }

\lstinputlisting[numbers=left, language=R, frame=single, tabsize=2, basicstyle=\footnotesize, showstringspaces=false]{esoph.R}

Die Varianzanalyse ergibt, dass mit einer Sicherheit von 89~\% ($\alpha = 0.11$) Alkohol- und Tabakkonsum zu einem erhöhten Risiko führt, an Speiseröhrenkrebs zu erkranken.

Ein lineares Modell ergibt mit einer Sicherheit von $p>99,9\%$, dass ein Zusammenhang zwischen dem Alter und dem Risiko, an Speiseröhrenkrebs zu erkranken, besteht.

\section{Julia}
\textit{Create a 2x4 two dimensional matrix with random floats in it and in the next step determine the biggest element.}

\texttt{zufall = rand(2, 4)}

\texttt{findmax(zufall)[1]}


\section{Julia}
\subsection{}
\textit{Create two matrices of the same layout and test if addition and subtraction of the matrix works as expected: C = A + B}

\texttt{A = [1 2 3 4; 4 3 2 1; 9 9 9 9; -5 3 4 0]}

\texttt{B = [4 3 2 1; -4 4 14 17.4; -9 9 -9 9; 0 -3 -4 0]}

\texttt{C = A + B}

\texttt{C = A - B}

\subsection{}
\textit{Now compare matrix multiplication either this way A * B and this way A .* B. Whats the difference?!}

Matrixmultiplikation:\\
\texttt{C = A * B}

Elementweise Multiplikation:\\
\texttt{C = A .* B}

\subsection{}
\textit{What about matrix division with ,,/'' or,,\textbackslash''?!}

\texttt{C = A / B} ergibt solch ein $C$, dass gilt $A = C \cdot B$.

\texttt{C = A $\backslash$ B} ergibt solch ein $C$, dass gilt $B = A \cdot C$.


\subsection{}
\textit{Create a 3x3 integer matrix A with useful numbers. Now try A+1, A-1, A*2, A/2.}

\texttt{A = [1 2 3; 4 3 2; -5 3 0]}

\texttt{A .+ 1}

\texttt{A .- 1}

\texttt{A .* 2}

\texttt{A ./ 2}

\subsection{}
\textit{Now multiply a 3x4 matrix with a suitable (4)vector. }

\texttt{randn(3, 4) * [-1, 0, -1, 0]}

oder

\texttt{randn(3, 4) * [-1; 0; -1; 0]}

Oder auch mit einem transponierten Zeilenvektor:

\texttt{randn(3,4) * transpose([-1 0 -1 0])}

\end{document}
