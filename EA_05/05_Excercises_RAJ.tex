\documentclass[11pt,a4paper]{article}
\usepackage[T1]{fontenc}
\usepackage[ngerman]{babel}
\usepackage{amsmath}
\usepackage{parskip}
%\usepackage{graphicx}
%\usepackage{listings}
\usepackage{hyperref}

%opening
\author{Simon Cholewa}
\title{5. R and Julia Excercise}

\hyphenation{
	Mo-tor-ü-ber-wach-ung 
}


\begin{document}

\maketitle

\section{R}
\textit{Write a program to guess a number in between 0 and 100!}

\textit{Hence the computer invents the number and the user = you tries to guess it! }


\section{R}
\textit{Analyse the \textbf{esoph} dataset. Can you derive some useful statements from it? Use data() to see all available datasets. }


\section{Julia}
\textit{Create a 2x4 two dimensional matrix with random floats in it and in the next step determine the biggest element.}

\section{Julia}
\subsection{}
\textit{Create two matrices of the same layout and test if addition and subtraction of the matrix works as expected: C = A + B}

\subsection{}
\textit{Now compare matrix multiplication either this way A * B and this way A .* B. Whats the difference?!}

\subsection{}
\textit{What about matrix division with ,,/'' or,,\textbackslash''?!}

\subsection{}
\textit{Create a 3x3 integer matrix A with useful numbers. Now try A+1, A-1, A*2, A/2.}

\subsection{}
\textit{Now multiply a 3x4 matrix with a suitable (4)vector. }

\end{document}
