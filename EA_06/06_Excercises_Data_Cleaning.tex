\documentclass[11pt,a4paper]{article}
\usepackage[T1]{fontenc}
\usepackage[ngerman]{babel}
\usepackage{amsmath}
\usepackage{parskip}
%\usepackage{graphicx}
\usepackage{listings}
\usepackage{hyperref}

%opening
\author{Simon Cholewa}
\title{6. Data Cleaning Excercise}

\hyphenation{
	Mo-tor-ü-ber-wach-ung 
}


\begin{document}

\maketitle

\section{dsm-beuth-edl-demodata-orig}

\textit{Have a look at the following dataset: \hyperref{https://github.com/edlich/eternalrepo/blob/master/DS-WAHLFACH/dsm-beuth-edl-demodata-dirty.csv}{}{}{dsm-beuth-edl-demodata-orig}}

\textit{Write a Python / Panda Script which 'cleans' this data set. Justify your actions in the respective notebook or python script you provide as a solution (link, file, kaggle repo, etc.).}

\textit{The original dataset does not necessarily have to be created. A proper strategy / good arguments are more important. Value: 5 points.}

\hyperref{https://raw.githubusercontent.com/grutzwurst/Data_Science/EA_06/EA_06/scripts/aufgabe1.py}{}{}{Script bei Github}

\lstinputlisting[numbers=left, language=Python, frame=single, tabsize=2, basicstyle=\footnotesize, showstringspaces=false]{scripts/aufgabe1.py}


\section{Kaggle Data Challenge}

\textit{Do the \textbf{Kaggle Data Challenge} by Rachel Tatman \hyperref{https://mailchi.mp/kaggle.com/5-day-data-challenge-data-cleaning-day-1?e=08afe8bf5d}{}{}{link}}

\begin{enumerate}
	\item \hyperref{https://www.kaggle.com/simoncholewa/data-cleaning-challenge-handling-missing-v-71e48a}{}{}{Handling missing}
	\item \hyperref{https://www.kaggle.com/simoncholewa/data-cleaning-challenge-scale-and-normalize-data}{}{}{Scale and Normalize Data}
	\item \hyperref{https://www.kaggle.com/simoncholewa/data-cleaning-challenge-parsing-dates}{}{}{Parsing Dates}
	\item \hyperref{https://www.kaggle.com/simoncholewa/data-cleaning-challenge-character-encodings}{}{}{Character Encodings}
	\item \hyperref{https://www.kaggle.com/simoncholewa/notebook2ae16dad09}{}{}{Inconsistent Data Entry}
\end{enumerate}




\end{document}
