\documentclass[11pt,a4paper]{article}
\usepackage[T1]{fontenc}
\usepackage[ngerman]{babel}
\usepackage{amsmath}
\usepackage{parskip}
%\usepackage{graphicx}
%\usepackage{listings}
%\usepackage{hyperref}
%\usepackage{float}

%opening
\author{Simon Cholewa}
\title{9. Excercises REG}

\hyphenation{
	Mo-tor-ü-ber-wach-ung 
}


\begin{document}

\maketitle

\section{Linear Regression}

\textit{Make up your own personal dataset and predict the regression curve. }


\section{Logistic Regression}

\textit{You are walking in the forrest and see an iris and measure:}

\textit{4.8,2.5,5.3,2.4}

\textit{Is this an Iris Virginica or not?}

\textit{The absolute minimum is to derive 4 probabilites for each feature. But it would be best to evaluate all 4 values in a (combined) equation to get one single probability for Iris Virginica! }


\end{document}
