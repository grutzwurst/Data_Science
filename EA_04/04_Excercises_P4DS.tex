\documentclass[11pt,a4paper]{article}
\usepackage[T1]{fontenc}
\usepackage[ngerman]{babel}
\usepackage{amsmath}
\usepackage{parskip}
\usepackage{graphicx}
\usepackage{listings}

%opening
\author{Simon Cholewa}
\title{4. Python 4 Data Science Excercise}

\hyphenation{
	Mo-tor-ü-ber-wach-ung 
}


\begin{document}

\maketitle

\section{Jupyter Notebook Setup}
\textit{Set up your Jupyter Notebook. Either by hand or by Anaconda or Google Colab! Play with it and send me e.g. a GitHub link or a PDF.
Do not send the *.ipynb !}
   
\section{101 NumPy Exercises}
\textit{Read and understand the first 25 Exercises from 101 NumPy Exercises. (send my any sloppy prove of a few solves ;-)}

\lstinputlisting[numbers=left, language=Python, frame=single, tabsize=2, basicstyle=\footnotesize, showstringspaces=false]{101.py}

\section{Jupyter Notebook Exercises}
\textit{Do the following exercise in a Jupyter Notebook (a GitHub Link would be the best):}

\textit{\begin{enumerate}
	\item Load the countries.csv directly via URL import into your panda data frame!
	\item Display some basic information as rows, columns and some basic statistical info.
	\item Show the last 4 rows of the data frame.
	\item Show all the row of countries who have the EURO
	\item Show only name and Currency in a new data frame
	\item Show only the rows/countries that have more than 2000 BIP (it is in Milliarden USD Bruttoinlandsprodukt)
	\item Select all countries where with inhabitants between 50 and 150 Mio
	\item Change BIP to Bip
	\item Calculate the Bip sum
	\item Calculate the average people of all countries
	\item Sort by name alphabetically
	\item Create a new data frame from the original where the area is changed as follows: all countries with > 1000000 get BIG and <= 1000000 get SMALL in the cell replaced!
\end{enumerate}}

\end{document}
