\documentclass[11pt,a4paper]{article}
\usepackage[T1]{fontenc}
\usepackage[ngerman]{babel}
\usepackage{amsmath}
\usepackage{parskip}
\usepackage{graphicx}
\usepackage{listings}
\usepackage{hyperref}

%opening
\author{Simon Cholewa}
\title{4. Python 4 Data Science Excercise}

\hyphenation{
	Mo-tor-ü-ber-wach-ung 
}


\begin{document}

\maketitle

\section{Jupyter Notebook Setup}
\textit{Set up your Jupyter Notebook. Either by hand or by Anaconda or Google Colab! Play with it and send me e.g. a GitHub link or a PDF.
Do not send the *.ipynb !}

In den folgenden beiden Abschnitten findet sich jeweils ein GitHub-Link.
   
\section{101 NumPy Exercises}
\textit{Read and understand the first 25 Exercises from 101 NumPy Exercises. (send my any sloppy prove of a few solves ;-)}

\href{https://github.com/grutzwurst/Data_Science/blob/EA_04/EA_04/101.py}{Skript mit einigen Übungen}

\lstinputlisting[numbers=left, language=Python, frame=single, tabsize=2, basicstyle=\footnotesize, showstringspaces=false]{101.py}

\section{Jupyter Notebook Exercises}
\textit{Do the following exercise in a Jupyter Notebook (a GitHub Link would be the best):}

Das Notebook mit den bearbeiteten Aufgaben findet sich hier: 

\href{https://github.com/grutzwurst/Data_Science/blob/EA_04/EA_04/Aufg4.ipynb}{Jupyter Notebook}

\end{document}
