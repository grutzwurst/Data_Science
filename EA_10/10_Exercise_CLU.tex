\documentclass[11pt,a4paper]{article}
\usepackage[T1]{fontenc}
\usepackage[ngerman]{babel}
\usepackage{amsmath}
\usepackage{parskip}
%\usepackage{graphicx}
%\usepackage{listings}
\usepackage{hyperref}
%\usepackage{float}

%opening
\author{Simon Cholewa}
\title{10. Instance Based Algs KNN Exercise}

\hyphenation{
	Mo-tor-ü-ber-wach-ung 
}


\begin{document}

\maketitle

\section{Exercise}

\textit{Implement KNN by hand for just 2 dimensions with normalization.}

\textit{This is easy because:}

\begin{enumerate}
	\item You normalize your data in another table
	\item You code a simple euclid distance function
	\item You take a point and calculate the distance to all points
	\item You take the list from above and sort it
	\item You aggregate by target variable
	\item you take the max to determine the targe class
\end{enumerate}

\textit{you are finished!}

\textit{Note: This is the only chance to implement a machine learning algorithm by hand and hence learn something from the ground up!} 


\section{Exercise}

\textit{In the logistic regression example, I gave you a new iris data:}

\textit{4.8,2.5,5.3,2.4}

\textit{Please classify this flower using KNN.}

\textit{Here is a good scikit example you can use: }

\begin{itemize}
	\item \hyperref{https://www.python-course.eu/k_nearest_neighbor_classifier.php}{}{}{Reference [1]}
	\item \hyperref{https://drive.google.com/open?id=1DnD_RRAZuanLlJSCmJjRbGtuloZVOirX}{}{}{Packed as a notebook [2]}
\end{itemize}


\end{document}
